\documentclass[11pt]{article}

\usepackage[utf8]{inputenc}
\usepackage{textcomp}
\usepackage{sectsty}
\usepackage{graphicx}
\usepackage{titlesec}
\usepackage[onehalfspacing]{setspace}
\usepackage{float}
\usepackage[ngerman]{babel}
\usepackage{fancyhdr}
\usepackage[a4paper, left=2.5cm, right=2cm, bottom=2cm, headheight=2cm]{geometry}
\usepackage{lipsum}
\usepackage{tocloft}

\setlength{\cftsecindent}{0em}
\setlength{\cftsubsecindent}{0em}
\setlength{\cftsubsubsecindent}{0em}
\setlength{\cftsecnumwidth}{4em}
\setlength{\cftsubsecnumwidth}{4em}
\setlength{\cftsubsubsecnumwidth}{4em}


\renewcommand{\sectionmark}[1]{\markright{#1}}
\pagestyle{fancy}
\fancyhf{}
\lhead{{\rightmark }} 
\fancyfoot[R]{\thepage}
\rhead{\includegraphics[scale=0.3]{Logo}}

\usepackage{graphicx}
\graphicspath{ {./Img/} }

\setcounter{secnumdepth}{4}


\setlength{\parindent}{0em}





\begin{document}

\begin{titlepage}
    \begin{center}
        \vspace*{1cm}
        
        \Huge
        \textbf{Infrastructure as Code}
        
        \Large
        \vspace{0.5cm}
        Die Rolle von Ansible für die Automatisierung und Dokumentation von 
        IT-Infrastrukturen
             
        \vspace{1.5cm}
 
        \textbf{Paul Herrmann}
 
        \vfill
                 
        \vspace{0.8cm}
      
        \includegraphics[width=1\textwidth]{Logo_Schule}
             
        reddo IT-SERVICE\\
        BSZ ET Dresden\\
        \today
             
    \end{center}
 \end{titlepage}

\thispagestyle{empty}
%\pagebreak
%\tableofcontents
%\thispagestyle{empty}
%\pagebreak
\setcounter{page}{1}

\section{Thema der Projektarbeit}
Das automatisierte Einrichten eines Webservers mithilfe von Ansible.

\section{Geplanter Bearbeitungszeitraum}
Beginn: 16.03.2024\\
Ende: 28.04.2024\\

\section{Projektbeschreibung}
Mein Projekt zum automatisierten Einrichten eines Webservers soll die 
Möglichkeiten der modernen Administrationstools
und Philosophien aufzeigen und gleichzeitig einen Vergleich zu klasischen Methodiken ziehen.\\

Meine Projektarbeit wird sich dabei auf die Philosophie des ''Infrastructure as Code'' st\"utzen und versucht
aufzuzeigen welche Vorteile diese birgt. Spezifischer soll der CI/CD Zyklus erklärt
und an einem simplen Beispiel aufgezeigt werden.\\

Ziel des praktischen Versuchs ist es, auf einer bestehenden, unkonfigurierten, virtuellen Linux-Maschine einen LEMP-Stack
zu installieren, der eine simple HTML-Seite darstellt. Ebenfalls sollte dieser nach einem Neustart die Webdienste
automatisch starten. Diese Einrichtung soll komplett über das Python basierte Automationstool Ansible stattfinden. 

\section{Projektumfeld}
Das Projekt findet im Umfeld der reddo IT-Service GmbH statt.

\section{Projektplan/Projektplan (einschließlich Zeitplanung)}

Konzeptionierung: 3h\\
- Ist-Analyse (1h)\\
- Soll-Betrachtung (1h)\\
- Lösungsfindung (1h)\\

Aufbauen der Testumgebung: 7h\\
- Einrichten des Hypervisors (3h)\\
- Erstellen des Webservers (1h)\\
- Erstellen und Einrichten des Automationsservers (3h)\\

Umsetzung: 15h\\
- Konfigurieren der Ansible Playbooks (10h)\\
- Pushen der Konfiguration (3h)\\
- Test des eingerichteten Servers (2h)\\

Dokumentation: 15h\\
- Schreiben der Projektarbeitsdokumentation (15h)\\

\section{Dokumentation zur Projektarbeit}
Ausarbeitung der Projektdokumentation in den beschriebenen Projektphasen, inkl.

- VM Konfigurationen\\
- Ansible Playbooks\\
- Dokumentationen zu den verwendeten Tools\\

\section{Themenbetreuer}
Name, Vorname: Osmalek, Hagen | Mail: hagen@reddo.de

\end{document}